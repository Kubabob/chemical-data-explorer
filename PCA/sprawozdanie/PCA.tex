\documentclass[12pt, a4paper]{article}

\renewcommand{\baselinestretch}{1.5}
\pagenumbering{arabic}
\pagestyle{plain}

\usepackage[margin=2cm]{geometry}
\usepackage{polski}
\usepackage[utf8]{inputenc}
\usepackage{indentfirst}
\usepackage[fleqn]{amsmath}
\usepackage{graphicx}
\usepackage{float}
\usepackage[tableposition=top]{caption}
\usepackage{hyperref}
\graphicspath{ {../figures/}}
\hypersetup{
    colorlinks=true,
    linkcolor=blue,
    filecolor=magenta,      
    urlcolor=blue,
    pdfpagemode=FullScreen,
    }

\author{Jakub Bożek}

\title{Sprawozdanie z PCA}
\begin{document}

\begin{titlepage}
    \centering
    {\LARGE \bfseries Sprawozdanie z PCA \par}
    \vspace{1cm}
    
    {\Large Techniki eksploracji danych wielowymiarowych \par}
    \vspace{2cm}
    
    {\LARGE \bfseries Jakub Bożek \par}
    \vspace{0.5cm}
    
    {\large 285665 \par}
    \vspace{0.5cm}
    
    {\large Bioinformatyka, II rok \par}
    \vspace{2cm}
    
    {\Large \today \par}
    
    \newpage
    \thispagestyle{empty}

    \tableofcontents
\end{titlepage}

\section{Wprowadzenie}

    \subsection{Cel ćwiczenia}

    Ćwiczenia z przedmiotu \it{Techniki eksploracji danych wielowymiarowych}

    \subsection{Krótkie wprowadzenie do metody PCA}

\section{Metodologia}

\section{Wyniki}

\section{Obserwacje}

\section{Dyskusja i wnioski}

\end{document}